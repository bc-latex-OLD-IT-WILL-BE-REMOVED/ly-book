\documentclass[12pt]{memoir}

\usepackage{lybook}

\usepackage[
    type={CC},
    modifier={by-nc-sa},
	version={4.0},
]{doclicense}


\newcommand\fakecontent{
	Bla, bla, bla, bla, bla, bla, bla, bla, bla
	bla, bla, bla, bla, bla, bla, bla, bla, bla
	bla, bla, bla, bla, bla, bla, bla, bla, bla
	la, bla, bla, bla, bla, bla, bla, bla, bla
	la, bla, bla, bla, bla, bla, bla, bla, bla
	\dots
}


\begin{document}

\buildfront
	{
		Le package \texttt{lybook}:
		\\
		taper des cours pour des lycéens
		\\
		{
			\footnotesize Code source disponible sur 
			\url{https://github.com/bc-latex/ly-book}.
		}
		\\
		{
			\footnotesize Version \texttt{0.0.0-beta} développée 
			et testée sur \macosxname{}.
		}
	}
	{Christophe BAL}
	{2019-09-05}
	{Documentation}
	{Chambéry (France)}


% ----------- %

\part{Mise en forme}

\chapter{Généralités}

\section{Pourquoi ?}

\subimport*{content/}{intro}


% ----------- %


\section{Avertissement}

\subimport*{content/}{classwarning}


% ----------- %


\section{Titre du manuel (obligatoire pour le moment)}

\subimport*{content/}{title}


% ----------- %


\section{Numérotation des parties, chapitres et sections}

\subimport*{content/}{sectionnumbering}


% ----------- %


\section{Écrire un théorème, une remarque, \dots}

\subimport*{content/}{proving}


% ----------- %


\section{Listes}

\subimport*{content/}{lists}


% ----------- %


\section{Multi-colonnes et traits verticaux}

\subimport*{content/}{multicols}


% ----------- %


\section{Entre des guillemets}

\subimport*{content/}{quote}


% ----------- %


\section{Des cadres}

\subimport*{content/}{frame}


% ----------- %


\part{Rédactions plus techniques -- TODO}

%\setcounter{section}{0}
%
%\section{Mathématiques}
%
%\subimport*{content/extra/}{math}
%
%
%% ----------- %
%
%
%\section{Algorithmique}
%
%\subimport*{content/extra/}{algo}
%
%
%% ----------- %
%
%
%\section{Codage informatique}
%
%\subimport*{content/extra/}{coding}
%
%
%% ----------- %
%
%
%\part{Packages utilisés -- Liste complète des dépendances}
%
%\subimport*{content/}{packages-used}

\end{document}