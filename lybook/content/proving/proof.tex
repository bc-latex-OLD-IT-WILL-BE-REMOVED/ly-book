L'environnement \verb+proof+ sert à la rédaction de preuves avec un éventuel sous-titre via un argument optionnel
\footnote{
	L'auteur assume cet anglicisime.
}
dont la numérotation sera celle du théorème, de la proposition, du corollaire ou de la remarque qui précède immédiatement la preuve. 


\medskip


Pour les preuves rédigées en classe avec les élèves, il suffit d'utiliser la commande \verb+\prooftodo+ qui évite d'avoir à taper un environnement \verb+proof+ de contenu vide. Cette macro admet elle aussi un éventuel sous-titre via un argument optionnel