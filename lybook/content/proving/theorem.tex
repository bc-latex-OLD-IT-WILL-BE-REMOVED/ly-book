Des environnements permettent de rédiger des théorèmes, des propositions, des corollaires, des exemples, des remarques, \dots avec un éventuel sous-titre via un argument optionnel
\footnote{
	On utilise juste ici les possibilités du package \texttt{amsmath}.
}. Voici la liste complète des environnements
\footnote{
	Noter que seul \texttt{notation} propose une version avec un pluriel.
}.

\begin{multicols-sep}{3}
% C
	\verb+conjecture+

	\verb+context+

	\verb+corollary+

% D
	\verb+demo+

% E
	\verb+example+

	\verb+exercice+
	
% L
	\verb+lemma+

% M
	\verb+methodology+

% N
	\verb+notation+
	
	\verb+notations+

% P
	\verb+proposition+

% R
	\verb+remark+

% S
	\verb+solution+

% T
	\verb+theorem+

% V
	\verb+vocabulary+
	
% Push up !
%	\vfill
%	\null
\end{multicols-sep}


\medskip


La numérotation est faite relativement aux sections et est commune à tous ces environnements
\footnote{
	Ce choix est le plus pratique pour un lecteur de documents sur papier.
}.
