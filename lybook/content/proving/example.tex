Voici un exemple de code, présenté sur deux colonnes
\footnote{
	Nous avons utilisé ici l'environnement \texttt{multicols-sep} proposé par le package \texttt{lybook}.
}
où \verb+\fakecontent+ est une macro définie dans le code de ce document.

\begin{frame-gene}{Code \LaTeX}
	\small
	\begin{multicols-sep}{2}
		\begin{verbatim}
	\begin{theorem}[Important]
    	\fakecontent
	\end{theorem}

	\begin{demo}[Unicité]
    	\fakecontent
	\end{demo}

	\begin{demo}[Existence]
    	\fakecontent
	\end{demo}
		\end{verbatim}


		\columnbreak


		\begin{verbatim}
	\begin{exercice*}{Mise en pratique}
    	\fakecontent
	\end{exercice*}

	\begin{solution*}{Les grandes lignes}
    	\fakecontent
	\end{solution*}

	\begin{example}
	    \fakecontent
	\end{example}

	\demotodo
		\end{verbatim}
	\end{multicols-sep}
\end{frame-gene}


% ---------------------- %


La mise en forme est la suivante.

\begin{frame-gene}{Rendu réel}
	\small

	\begin{theorem}[Important]
    	\fakecontent
	\end{theorem}

	\begin{demo}{Unicité}
    	\fakecontent
	\end{demo}

	\begin{demo}{Existence}
    	\fakecontent
	\end{demo}
	
	\begin{exercice*}{Mise en pratique}
    	\fakecontent
	\end{exercice*}

	\begin{solution*}{Les grandes lignes}
    	\fakecontent
	\end{solution*}

	\begin{example}
	    \fakecontent
	\end{example}

	\demotodo
\end{frame-gene}
