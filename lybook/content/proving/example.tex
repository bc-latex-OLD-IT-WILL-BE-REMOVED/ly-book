Voici un exemple de code, présenté sur deux colonnes
\footnote{
	Nous avons utilisé ici l'environnement \texttt{multicols-sep} proposé par le package \texttt{lybook}.
}
où \verb+\fakecontent+ est une macro définie dans le code de ce document.

\begin{frame-gene}[Code \LaTeX]
\small
\begin{multicols-sep}{2}
	\begin{verbatim}
\begin{theorem}[Important]
    \fakecontent
\end{theorem}

\prooftodo[Juste l'unicité]

\begin{proposition}
    \fakecontent
\end{proposition}

\begin{corollary}
    \fakecontent
\end{corollary}\end{verbatim}

	\columnbreak
	
	\begin{verbatim}
\begin{lemma}
    \fakecontent
\end{lemma}

\prooftodo

\begin{remark}
    \fakecontent
\end{remark}

\begin{proof}
    \fakecontent
\end{proof}\end{verbatim}
\end{multicols-sep}
\end{frame-gene}


La mise en forme est la suivante.

\begin{frame-gene}[Rendu réel]
\small
\begin{theorem}[Important]
    \fakecontent
\end{theorem}

\prooftodo[Juste l'unicité]

\begin{proposition}
    \fakecontent
\end{proposition}

\begin{corollary}
    \fakecontent
\end{corollary}

\begin{lemma}
    \fakecontent
\end{lemma}

\prooftodo

\begin{remark}
    \fakecontent
\end{remark}

\begin{proof}
    \fakecontent
\end{proof}
\end{frame-gene}
