Voici un exemple de code, présenté sur deux colonnes
\footnote{
	Nous avons utilisé ici l'environnement \texttt{multicols-sep} proposé par le package \texttt{lybook}.
}
où \verb+\fakecontent+ est une macro définie dans le code de ce document.

\begin{frame-gene}[Code \LaTeX]
	\small
	\begin{multicols-sep}{2}
		\begin{verbatim}
	\begin{theorem}[Important]
    	\fakecontent
	\end{theorem}

	\demotodo[Juste l'unicité]

	\begin{vocabulary}
    	\fakecontent
	\end{vocabulary}

	\begin{demo}
	    Attention ! Ce n'est pas
	    sémantiquement correct.
	\end{demo}

	\begin{methodology}
	    \fakecontent
	\end{methodology}
		\end{verbatim}


		\columnbreak


		\begin{verbatim}
	\begin{exercice}[Mise en pratique]
    	\fakecontent
	\end{exercice}

	\begin{solution}[Les grandes lignes]
    	\fakecontent
	\end{solution}

	\begin{example*}
	    \fakecontent
	\end{example*}

	\demotodo*
		\end{verbatim}
	\end{multicols-sep}
\end{frame-gene}


% ---------------------- %


La mise en forme est la suivante.

\begin{frame-gene}[Rendu réel]
	\small
	\begin{theorem}[Important]
    	\fakecontent
	\end{theorem}

	\demotodo[Juste l'unicité]

	\begin{vocabulary}
    	\fakecontent
	\end{vocabulary}

	\begin{demo}
    	Attention ! Ce n'est pas sémantiquement correct.
	\end{demo}

	\begin{methodology}
	    \fakecontent
	\end{methodology}

	\begin{exercice}[Mise en pratique]
    	\fakecontent
	\end{exercice}

	\begin{solution}[Les grandes lignes]
    	\fakecontent
	\end{solution}

	\begin{example*}
	    \fakecontent
	\end{example*}

	\demotodo*
\end{frame-gene}
