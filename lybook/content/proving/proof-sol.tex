L'environnement \verb+solution+ sert à la rédaction de solutions d'exercices avec un éventuel sous-titre via un argument optionnel. On prendra garde que la numérotation sera celle de tout environnement
\footnote{
	Y compris un environnement qui n'est pas un exercice !
}
qui précède immédiatement la solution et qui a une numérotation commune avec celle des théorèmes comme par exemple les propositions ou les remarques bien que ce ne soit pas pour des exercices.


\medskip


L'environnement \verb+demo+ est similaire sert à l'environnement \verb+solution+ mais pour rédiger des preuves
\footnote{
	L'auteur assume cet anglicisme.
}.
Pour les preuves rédigées en classe avec les élèves, il suffit d'utiliser la commande \verb+\demotodo+ qui évite d'avoir à taper un environnement \verb+demo+ de contenu vide. Cette macro admet elle aussi un éventuel sous-titre via un argument optionnel.