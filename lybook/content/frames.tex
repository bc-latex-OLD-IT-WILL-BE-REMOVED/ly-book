Pour bien séparer certains contenus particuliers comme du code par exemple, il est possible d'utiliser des cadres avec ou sans titre
\footnote{
    Merci au package \texttt{tcolorbox}.
}.
Voici un exemple de code où \verb+\fakecontent+ est une macro définie dans le code de ce document.

\begin{frame-gene}{Code \LaTeX}
\small
\begin{verbatim}
\begin{frame-gene}{Un titre personnalisable}
    \fakecontent
\end{frame-gene}

\begin{frame-gene}{}  % ATTENTION ! Le titre est un argument obligatoire.
    Pas de titre pour moi. Merci !
\end{frame-gene}
\end{verbatim}
\end{frame-gene}


\medskip


Nous mettons directement la mise en forme obtenue ci-dessous.

\begin{frame-gene}{Un titre personnalisable}
    \fakecontent
\end{frame-gene}

\begin{frame-gene}{}
    Pas de titre pour moi. Merci !
\end{frame-gene}


\medskip

On peut contrôler la largeur relativement à la largeur de la ligne via un argument optionnel. Dans le code précédent, l'utilisation de \verb+\begin{frame-gene}[.5]{...}+ au lieu de \verb+\begin{frame-gene}{...}+, on obtient ce qui suit.

\begin{frame-gene}[.5]{Un titre personnalisable}
    \fakecontent
\end{frame-gene}

\begin{frame-gene}[.5]{}
    Pas de titre pour moi. Merci !
\end{frame-gene}
