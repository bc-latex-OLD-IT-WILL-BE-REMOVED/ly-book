Pour bien séparer certains contenus particuliers comme du code par exemple, il est possible d'utiliser des cadres avec ou sans titre
\footnote{
    Merci au package \texttt{tcolorbox}.
}.
Voici un exemple de code où \verb+\fakecontent+ est une macro définie dans le code de ce document.

\begin{frame-gene}[Code \LaTeX]
\small
\begin{verbatim}
\begin{frame-gene}[Un titre personnalisable]
    \fakecontent
\end{frame-gene}

\begin{frame-gene}
    Pas de titre pour moi. Merci !
\end{frame-gene}
\end{verbatim}
\end{frame-gene}



\medskip

Nous mettons directement la mise en forme obtenue ci-dessous.

\begin{frame-gene}[Un titre personnalisable]
    \fakecontent
\end{frame-gene}

\begin{frame-gene}
    Pas de titre pour moi. Merci !
\end{frame-gene}
